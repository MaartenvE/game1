

\documentclass[a4paper,titlepage]{scrartcl}

\usepackage[english]{babel}
\usepackage[utf8]{inputenc}
\usepackage{amsmath}
\usepackage{graphicx}
\usepackage[colorinlistoftodos]{todonotes}
\usepackage{hyperref}
\usepackage{fancyref}
\usepackage{graphicx}

\title{Test plan}
\subtitle{TI2805 Context Project, Computer Games, Team Cubed (\textit{Building Blocks})}

\author{Gert Spek \\ gspek \\ 4216806 \and Niels Bakker \\ ncbakker \\ 4161394 \and Jehan da Camara \\ jdacamara \\ 4207858 \and Maarten van Elsas \\ mvanelsas \\ 4176898 \and Leon Helsloot \\ lhelsloot \\ 4235991}

\date{\today}

\begin{document}
\maketitle

\begin{abstract}
This document describes the test plan of Team Cubed (group 1) of the Computer Games Context Project. It describes which test sessions we would like to organise, which users we would involve in these test sessions and what we want to achieve with these sessions.
\end{abstract}

\newpage
\tableofcontents

\section{Testing sessions}
The target audience of the \textit{Building Blocks} game consists of anyone who is a bit competitive and can handle a smartphone. This includes teenagers and students, but also many children and adults and even some elderly. Naturally, as developers of the \textit{Building Blocks} smartphone game, we know best how the game is built and what its features are. But since the idea is not for us to play solely with one another, we will organise two sessions where we would get other users not familiar with the game and let them interact with the game and give feedback, so we can try to improve the game based on their opinions.

The first test session we will organize will take place on \textbf{Tuesday, June 10}, before the beta release. Since the product will probably still contain many bugs at this point, this test session will be aimed at a more forgiving audience, namely university students. We will observe which actions the users take when we don't explain the game to them, so we can determine which game elements are intuitive and which elements either require an explanation or should be redesigned. The session will take place at the Drebbelweg, building 35, during the break.

The second test session will be held \textbf{Tuesday, June 17}, exactly one week before the final release. This test session will be organized at the Science Centre if possible, where a wide range of people can play the game. During the session, we want to determine what unexpected behaviour other people may show and which bugs this might cause. This gives us a week to fix problems before the final release.


\section{Approach towards testing}
Our approach towards testing is pretty straightforward. We bring a group of people together, explain the game very briefly or not even at all and see how they interact. But we do have some criteria for the users that we would be using. Since we are on a low budget we would gather a group of roughly 8 people, all of whom have an android device, and connect them to a server and see how it goes. We would prefer that these people would not be Computer Science students because for computer scientists some things are more straightforward than for a normal person.

\section{Game testing component breakdown structure}
What we would like to achieve in these test sessions is to see how the user interaction is between the users with each other and with the game. Should actions which we thought to be straightforward turn out to be more complicated than expected for the testers, we could improve the gameplay for our future users. We also want to uncover bugs caused by behaviour we had not anticipated.

\end{document}